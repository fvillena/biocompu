\documentclass{article}

\usepackage[utf8]{inputenc}
\usepackage[spanish]{babel}
\usepackage{hyperref}
\usepackage{listings}

\title{Ajuste de modelos}
\author{Fabián Villena y Felipe Arias}
\date{Julio 2023}

\begin{document}

\maketitle

Se le solicita ajustar (o entrenar) un modelo de aprendizaje automático para predecir etiquetas sobre ejemplos nuevos.

En particular, trabajaremos con el \textit{Pima Indians Diabetes Dataset}. Este conjunto de datos tiene como objetivo predecir si un paciente tiene o no tiene diabetes en base a una serie de características. La muestra consiste en datos de mujeres de al menos 21 años de descendencia indígena del grupo \textit{Pima}. Específicamente, existen varios atributos médicos y una variable objetivo, la columna \textit{Outcome}. Esta columna alcanza un valor igual a 1 cuando la paciente posee diabetes, mientras que es 0 cuando no posee la enfermedad.

Deben realizar un flujo clásico al momento de crear modelos de aprendizaje automático.

El conjunto de datos se encuentra en la siguiente dirección:

\begin{center}
    \url{https://github.com/fvillena/biocompu/blob/2023/data/diabetes.csv}
\end{center}

\section*{Preguntas}

Responda las siguientes preguntas en un \textit{Jupyter Notebook} con código desarrollado en el lenguaje de programación Python.

\begin{enumerate}
	\item ¿Cuántas instancias y cuántos atributos contiene el conjunto de datos?
    \item ¿Cuál es la frecuencia de ejemplos en la variable objetivo \textit{Outcome}?
    \item ¿Cuántos pacientes mayores a 40 años padecen de diabetes?
    \item Separe el conjunto de datos en un subconjunto de entrenamiento y uno de prueba
    \item Ajuste un modelo de clasificación con el subconjunto de entrenamiento.
    \item Prediga utilizando el modelo ajustado sobre el subconjunto de prueba.
    \item Calcule la métrica de \textit{accuracy} del modelo sobre el conjunto de prueba.
\end{enumerate}

\end{document}
